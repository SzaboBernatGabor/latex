\documentclass{article}
\usepackage{amsthm}
\usepackage{float}
\usepackage{babel}
\usepackage{hulipsum}
\usepackage{sidecap}
\usepackage{listings}
\usepackage{algpseudocode}
\usepackage{algorithm}
\newfloat{forraskod}{hbt}{verbatim}
\newtheorem*{megj}{Megjegyzés}
\theoremstyle{plain}
\newtheorem{tet}{Tétel}
\newtheorem{lem}[tet]{Lemma}
\theoremstyle{definition}
\newtheorem{defi}{Definíció}
\theoremstyle{remark}
\newtheorem{fel}{Feladat}[section]
\algblockdefx{DoWhile}{EndDoWhile}{\textbf{do}}{\textbf{while}}
\algblockdefx{Cases}{EndCases}{\textbf{cases}}{\textbf{end cases}}
\algnewcommand\algorithmicwhen{\textbf{when}}
\algnewcommand\algorithmicotherwise{\textbf{otherwise}}
\algnewcommand\algorithmicendwhen{\textbf{end when}}
\algnewcommand\When[1]{\item[\algorithmicwhen\ #1:]}
\algnewcommand\Otherwise{\item[\algorithmicotherwise:]}
\begin{document}
\listof{forraskod}{Forráskód}
\listofalgorithms
\begin{tet}
Első tétel
\end{tet}
\begin{proof}
Bizonyítása
\end{proof}
\begin{defi}
Első definícó
\end{defi}
\begin{tet}
Második tétel
\end{tet}
\begin{defi}
Második definíció
\end{defi}
\begin{lem}
Lemma
\end{lem}
\section{Feladatsor 1}
\begin{fel}
Első feladat
\end{fel}
\begin{fel}
Második feladat
\end{fel}
\section{Feladatsor 2}
\begin{fel}
Harmadik feladat
\end{fel}
\begin{fel}
Negyedik feladat
\end{fel}
\begin{fel}
Ötödik feladat
\end{fel}
\begin{tet}
Tétel az ötödik feladathoz
\end{tet}
\begin{defi}
Definíció az ötödik feladathoz
\end{defi}
\begin{megj}
Megjegyzés
\end{megj}
\verb|\LaTeX \LaTeX \LaTeX|

\verb|\newtheorem{tet}{Tétel}|

\hulipsum[1]
\begin{forraskod}
\caption{Forráskód}
\label{forraskod}
\begin{verbatim}
\begin{itemize}
\item Ez egy lista
\end{itemize}
\end{verbatim}
\end{forraskod}
\newpage
\lstinputlisting[language=C, tabsize=5, numbers=left, stepnumber=4, float, label=c_kod]{main.c}

\begin{algorithm}
\caption{Gyorsrendezés}
\label{gyorsrendezes}
\begin{algorithmic}[2]
\Procedure{Quicksort}{@A,a,b}
\Require A írható tömb
\Require 1 <= a <= b <= Hossz[A] indexek
\Ensure a-b indextartományt rendezzük
\If{a=b}
\Return A
\Else
\Call{Feloszt}{@A,a,b,A(a),@q}
\Call{Quicksort}{@A,a,q}
\Call{Quicksort}{@A,q+1,b}
\Return A
\EndIf
\EndProcedure
\end{algorithmic}
\end{algorithm}
\begin{algorithmic}
\DoWhile
\State i+1
\EndDoWhile{$i < x$}
\end{algorithmic}
\begin{algorithmic}
\State $x \gets$ input value
\Cases
\When{$x > 0$}
\State Végrehajtandó művelet, ha $x > 0$
\When{$x < 0$}
\State Végrehajtandó művelet, ha $x < 0$
\Otherwise
\State Végrehajtandó művelet, ha egyéb esetben
\EndCases
\end{algorithmic}
\end{document}