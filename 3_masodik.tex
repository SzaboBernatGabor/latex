\documentclass{article}
\usepackage[inline]{enumitem}
\usepackage{hulipsum}
\usepackage{graphicx}
\usepackage{subcaption}
\usepackage{array}
\usepackage{tabularx}
\usepackage{colortbl}
\usepackage{xcolor}
\begin{document}
\listoffigures
\listoftables
\begin{enumerate}
\begin{itemize*}[label=\textbullet,%
itemjoin*={\hspace{0.1em} és }]
\item egy
\item ketto
\item harom	
\end{itemize*}
\end{enumerate}
\newlist{mynested}{enumerate}{5}
\setlist[mynested,1]{label=\arabic*.}
\setlist[mynested,2]{label=\alph*.}
\setlist[mynested,3]{label=\roman*.}
\setlist[mynested,4]{label=\Alph*.}
\setlist[mynested,5]{label=\Roman*.}
\begin{mynested}
    \item Az első szint 1.
    \begin{mynested}
        \item Második szint a)
        \begin{mynested}
            \item Harmadik szint i)
            \begin{mynested}
                \item Negyedik szint A.
                \begin{mynested}
                    \item Ötödik szint I.
                \end{mynested}
            \end{mynested}
        \end{mynested}
    \end{mynested}
\end{mynested}
\hulipsum [1]
\begin{mynested}[resume, label=\roman*,ref=(\roman*)]
\item folytat
\begin{mynested}[label=\roman*,ref=(\roman*)]
\item[] szamozatlan
\end{mynested}
\end{mynested}
\begin{description}[style=sameline]
    \item[] \hulipsum[1]
    \item[slanted] \hulipsum[1]
    \item[hosszu cimke] \hulipsum[1]
\end{description}
\begin{figure}[b]
  \centering
  \begin{subfigure}{0.45\textwidth}
    \includegraphics[width=5cm]{kep.jpg}
    \caption{elso kep}
    \label{fig:subfig1}
  \end{subfigure}
  \begin{subfigure}{0.45\textwidth}
    \includegraphics[angle=180, width=5cm]{kep.jpg}
    \caption{masodik kep}
    \label{fig:subfig2}
  \end{subfigure}
  \caption{ket kulon kep}
  \label{fig:mainfig}
\end{figure}
\begin{tabular}{l||r|c|l|}
{} & egy & ketto & harom\\
\hline
\hline
Hello & negy & ot & hat\\
vilag! & {} & {} & {}\\
\cline{2-4}
{} & het & nyolc & kilenc\\
\cline{2-2} \cline{4-4}
lorum & tiz & {} & tizenketto\\
ipse & {} & {} & {}\\
\hline
\end{tabular}
\\
\\
\\
\begin{tabular}{r|c|l}
egy & ketto & harom\\
\hline
\rowcolor{yellow}
negy & ot & hat\\
\rowcolor{blue}
het & nyolc & kilenc\\
\rowcolor{yellow}
tiz & {} & tizenketto\\
\end{tabular}
\\
\\
\\
\begin{tabular}{r|cl}
\cellcolor{black}\textcolor{green}{egy} & ketto & harom\\ %nem fogadja el a gray színt :(
\arrayrulecolor{green}
\hline	
\cellcolor{black}\textcolor{white}{negy} & ot & hat\\
\cellcolor{black}\textcolor{white}{het} & nyolc & \cellcolor{green}kilenc\\
\cellcolor{black}\textcolor{white}{tiz} & {} & tizenketto\\
\end{tabular}
\end{document}